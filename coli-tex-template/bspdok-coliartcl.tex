\documentclass[12pt,        % standard font size
  english,ngerman,          % english secondary, german primary
  paper=a4,                 % standard paper size
  captions=tablesignature,  % captions below tables
  listof=numbered,          % including lists of... in the ToC
  bibliography=totoc,       % references in ToC
  headings=small,           % size of headings
  headinclude=false,        % don't include page head in page layout
  footinclude=false,        % don't include page foot in page layout
  parskip=half-,            % space between paragraphs, no indentation
  oneside,                  % one-sided print
  %twocolumn,
  DIV=12                    %
]{styles/coliartcl}


% most setup is done in a file of its own
%%%%%%%%%%%%%%%%%%%%%%%%%%%%%%%%%%%%%%%%%%%%%%%%%%%%%%%%%%%%
% Now we can start to load the packages needed
%%%%%%%%%%%%%%%%%%%%%%%%%%%%%%%%%%%%%%%%%%%%%%%%%%%%%%%%%%%%

%%%%%%%%%% Testing  %%%%%%%%%%
% provides Lorem Ipsum blindtext
\usepackage{lipsum}


%%%%%%%%%% General      %%%%%%%%%%
% koma comes with more advanced handling of tocs, scrhack
% makes other packages like listings work with that
% TODO this might cause problems with updated packages
\usepackage{scrhack}
% we want to support both pdfLaTeX and XeLaTeX
% TODO consider luaTeX as well
\usepackage{ifxetex}
% language settings, output-encoding
% localisation (languages set in document directive):
\usepackage{babel}
% % TODO context sensitive quotes, investigate for biblatex
% \usepackage{csquotes}
% input encoding to support diacritics
\usepackage[utf8]{inputenc}
% output encoding:
\usepackage[T1]{fontenc}
% acronym list
\usepackage[printonlyused]{acronym}


%%%%%%%%%% Look & Feel  %%%%%%%%%%
% define colors as you need them
\usepackage{color}
% nicer headings/footer:
\usepackage{scrpage2}
% typeset URLs nicely:
\usepackage[hyphens]{url}
% extended table environment:
\usepackage{tabularx}
% % colours in tables:
% \usepackage{colortbl}
% % booktabs for nice looking tables
% \usepackage{booktabs}
% used for adapting list environment, e.g. for glossaries:
\usepackage{enumitem}
% include source code listings in the text:
\usepackage{listings}

\usepackage{multicol}

%%%%%%%%%% Graphics     %%%%%%%%%%
% we need to be able to include graphics:
\usepackage{graphicx}
% we use colours in tables etc.:
\usepackage[dvipsnames,usenames]{xcolor}
% TikZ is cool for diagrams:
\usepackage{tikz}
% these 2 packages can be used to generate
% graphs & tables from tabular data:
\usepackage{pgfplots}
\usepackage{pgfplotstable}
% pinning to specific pgfplots version
\pgfplotsset{compat=1.8}
% make text flow around images
% \usepackage{picins}

%%%%%%%%%% Color Definitions %%%%%
% color definitions for listings (example in java)
\definecolor{pblue}{rgb}{0.13,0.13,1}
\definecolor{pgreen}{rgb}{0,0.5,0}
\definecolor{pred}{rgb}{0.9,0,0}
\definecolor{pgrey}{rgb}{0.46,0.45,0.48}

% color definitions corporate design vzg and coli-conc
\definecolor{vzgheadline}{RGB}{0,102,102}
\definecolor{vzgpetrol}{RGB}{47,140,144}
\definecolor{vzgblue}{RGB}{84,128,187}
\definecolor{coliorange}{HTML}{C14E00}
\definecolor{coliblack}{HTML}{393939}
\definecolor{coligrey}{HTML}{E3E3DF}

%%%%%%%%%% lstset Definitions %%%%%
\lstset{language=Java,
  frame=single,
  showspaces=false,
  showtabs=false,
  breaklines=true,
  showstringspaces=false,
  breakatwhitespace=true,
  captionpos=b,
  nolol=false,
  numbers=left,
  numbersep=5pt,
  numberstyle=\tiny\color{pgrey},
  commentstyle=\color{pgreen},
  keywordstyle=\color{pblue},
  stringstyle=\color{pred},
  basicstyle=\ttfamily,
  moredelim=[il][\textcolor{pgrey}]{\$\$},
  moredelim=[is][\textcolor{pgrey}]{\%\%}{\%\%}
}

%%%%%%%%%% Bib          %%%%%%%%%%
% allows us to use author-year styles:
\usepackage[round,colon,sort]{natbib}
% \usepackage{natbib}
% allows to have a separate bibliography for weblinks (or more):
\usepackage{multibib}

%%%%%%%%%% Other        %%%%%%%%%%
% % is already loaded automatically:
% \usepackage{calc}
% % do we want an index?
% \usepackage{makeidx}
% include hyperlinks in PDF and much more:
% FIXME Chose between coloured or black
\usepackage[
  colorlinks=true,
  linkcolor=coliorange,
  menucolor=coliorange,
  citecolor=coliorange,
  urlcolor=coliorange,
]{hyperref}
\hypersetup{linktoc=all}

%%%%%%%%%%%%%%%%%%%%%%%%%%%%%%%%%%%%%%%%%%%%%%%%%%%%%%%%%%%%
%   lists of... and captions
%%%%%%%%%%%%%%%%%%%%%%%%%%%%%%%%%%%%%%%%%%%%%%%%%%%%%%%%%%%%

% for listings, we define the names used in the captions
% and in the list of listings
% we also take the -verzeichnis out of the list of
% figures and list of tables
% TODO differences in defined languages/babel; Text formating
\AtBeginDocument{%
  \makeatletter
    \addto\captionsngerman{%
      \def\lstlistingname{Quelltext}%
      \def\lstlistlistingname{Quelltexte}%
%       \renewcommand{\listfigurename}{\normalfont\rmfamily Abbildungen}%
%       \renewcommand{\listtablename}{\normalfont\rmfamily Tabellen}%
      \renewcommand{\listfigurename}{Abbildungen}%
      \renewcommand{\listtablename}{Tabellen}%
    }%
  \makeatother
}
% % for other than KOMA classes, the following may have
% % to be used instead:
% \renewcommand*{\lstlistlistingname}{Quelltexte}
% \renewcommand*{\lstlistingname}{Quelltexte}

% captions should be set in a small font size
\addtokomafont{caption}{\footnotesize}
% \addtokomafont{captionlabel}{\bfseries}

%%%%%%%%%%%%%%%%%%%%%%%%%%%%%%%%%%%%%%%%%%%%%%%%%%%%%%%%%%%%
%   General changes to formatting
%%%%%%%%%%%%%%%%%%%%%%%%%%%%%%%%%%%%%%%%%%%%%%%%%%%%%%%%%%%%

% this is the KOMA-way of changing titles.
%\setkomafont{sectioning}{\rmfamily\bfseries}
\setkomafont{disposition}{\color{coliorange}\rmfamily\bfseries}
  
% we want an url to be printed in the standard font
\urlstyle{same}
% in the following file most of the necessary input has to be made
%%%%%%%%%%%%%%%%%%%%%%%%%%%%%%%%%%%%%%%%%%%%%%%%%%%%%%%%%%%%
%   General information input part
%%%%%%%%%%%%%%%%%%%%%%%%%%%%%%%%%%%%%%%%%%%%%%%%%%%%%%%%%%%%

% FIXME enter document title
\newcommand{\colititle}{Coli-conc Report Template}

% FIXME enter the subject(s) divided by comma
\newcommand{\colisubject}{Template, VZG, Coli-conc, Report}

% FIXME real date of delivery. \today is todays date as dd. MM yyyy
\newcommand{\colidate}{\today}
% \newcommand{\colidate}{1. April 1970}

% FIXME enter the e-mail-address
\newcommand{\colimail}{coli-conc@gbv.de}


%%%%%%%%%%%%%%%%%%%%%%%%%%%%%%%%%%%%%%%%%%%%%%%%%%%%%%%%%%%%
%   For coli-documentation and coli-report only
%%%%%%%%%%%%%%%%%%%%%%%%%%%%%%%%%%%%%%%%%%%%%%%%%%%%%%%%%%%%

% FIXME (un)mark the version that is needed
% \newcommand{\colireportlang}{No. }
\newcommand{\colireportlang}{Nr. }

% FIXME enter the report number
\newcommand{\colireportno}{11}


%%%%%%%%%%%%%%%%%%%%%%%%%%%%%%%%%%%%%%%%%%%%%%%%%%%%%%%%%%%%
%   For coli-article and coli-report only
%%%%%%%%%%%%%%%%%%%%%%%%%%%%%%%%%%%%%%%%%%%%%%%%%%%%%%%%%%%%

% Number of authors and working addresses has to correspond.
% The tex-file has to be adjusted accordingly.
% FIXME enter the document author(s)
\newcommand{\coliauthorone}{Mustermann, Max}
\newcommand{\coliauthortwo}{Musterfrau, Martha}
\newcommand{\coliauthorthree}{Bla}
\newcommand{\coliauthorfour}{Blubb}


%%%%%%%%%%%%%%%%%%%%%%%%%%%%%%%%%%%%%%%%%%%%%%%%%%%%%%%%%%%%
%   For coli-article only
%%%%%%%%%%%%%%%%%%%%%%%%%%%%%%%%%%%%%%%%%%%%%%%%%%%%%%%%%%%%

% FIXME enter the author(s) working address(es)
% has to be the same amount as authors in the part before!
\newcommand{\coliaddressone}{Verbundzentrale des GBV (VZG), Göttingen}
\newcommand{\coliaddresstwo}{Department of Computer Science, University of Wherever}
\newcommand{\coliaddressthree}{Department of Biology, University of Wherever}
\newcommand{\coliaddressfour}{Department of Dingsda, University of Wherever}


%%%%%%%%%%%%%%%%%%%%%%%%%%%%%%%%%%%%%%%%%%%%%%%%%%%%%%%%%%%%
%   For coli-documentation only
%%%%%%%%%%%%%%%%%%%%%%%%%%%%%%%%%%%%%%%%%%%%%%%%%%%%%%%%%%%%

% FIXME enter the document author(s)
\newcommand{\coliauthor}{Mustermann, Max und Musterfrau, Martha}

% FIXME (un)mark the needed text type
% \newcommand{\colitexttype}{Technical Documentation}
\newcommand{\colitexttype}{Technische Dokumentation}

% FIXME (un)mark what you need/want
% text type + enumeration
% \newcommand{\colidokfooter}{\colitexttype~\colireportlang~\colireportno}
% only text type
% \newcommand{\colidokfooter}{\colitexttype}
% only title
\newcommand{\colidokfooter}{\colititle}


%%%%%%%%%%%%%%%%%%%%%%%%%%%%%%%%%%%%%%%%%%%%%%%%%%%%%%%%%%%%
%   For coli-report only
%%%%%%%%%%%%%%%%%%%%%%%%%%%%%%%%%%%%%%%%%%%%%%%%%%%%%%%%%%%%

% FIXME enter the author(s) e-mail-address(es)
% has to be the same amount as authors specified in the section
% For coli-article and coli-report only
\newcommand{\colimailone}{max.mustermann@vzg.de}
\newcommand{\colimailtwo}{martha.musterfrau@vz.de}
\newcommand{\colimailthree}{bla@vzg.de}
\newcommand{\colimailfour}{blubb@vzg.de}

% FIXME enter the affiliation
\newcommand{\coliinstitute}{Verbundzentrale des GBV (VZG), Göttingen}

% FIXME enter the DOI address
\newcommand{\colidoi}{<DOI Address>}

% FIXME enter the used licence
\newcommand{\colilicence}{CC-BY-SA}

% nothing to enter here!
\newcommand{\colifooter}{Coli-conc Report\colireportlang~\colireportno}


%%%%%%%%%%%%%%%%%%%%%%%%%%%%%%%%%%%%%%%%%%%%%%%%%%%%%%%%%%%%
% lstset Definitions
%%%%%%%%%%%%%%%%%%%%%%%%%%%%%%%%%%%%%%%%%%%%%%%%%%%%%%%%%%%%
% FIXME define for the language you are really using
% Java is just an example, but usable if needed
\lstset{language=Java,
  frame=single,
  showspaces=false,
  showtabs=false,
  breaklines=true,
  showstringspaces=false,
  breakatwhitespace=true,
  captionpos=b,
  nolol=false,
  numbers=left,
  numbersep=5pt,
  numberstyle=\tiny\color{pgrey},
  commentstyle=\color{pgreen},
  keywordstyle=\color{pblue},
  stringstyle=\color{pred},
  basicstyle=\ttfamily,
  moredelim=[il][\textcolor{pgrey}]{\$\$},
  moredelim=[is][\textcolor{pgrey}]{\%\%}{\%\%}
}


%%%%%%%%%%%%%%%%%%%%%%%%%%%%%%%%%%%%%%%%%%%%%%%%%%%%%%%%%%%%
% Start with the document
%%%%%%%%%%%%%%%%%%%%%%%%%%%%%%%%%%%%%%%%%%%%%%%%%%%%%%%%%%%%

\begin{document}

% FIXME select used language    
%\selectlanguage{english} % switching to english
\selectlanguage{ngerman} % switching to german

% the paper's title here
\title{\colititle}

% Place the author information here.  Please hand-code the contact
% information and notecalls; do *not* use \footnote commands.  Let the
% author contact information appear immediately below the author names
% as shown.
% The number of authors and addresses has to be equal to those, specified in
%% the input.tex.
\author
	{
		\coliauthorone,$^{1\ast}$ 
    	\coliauthortwo,$^{1}$ 
    	\coliauthorthree$^{2}$
    	\\
		\\
		\normalsize{$^{1}$
			\coliaddressone
		}
		\\
		\normalsize{
			\coliaddresstwo
		}
		\\
		\normalsize{$^{2}$
			\coliaddressthree
		}
		\\
		\\
		\normalsize{$^\ast$To whom correspondence should be addressed. E-Mail: \colimail}
	}

% Include the date command, but leave its argument blank.
\date{
	\colidate
}
%%%%%%%%%%%%%%%%% END OF PREAMBLE %%%%%%%%%%%%%%%%

% Make the title.
\maketitle

% abstract
\begin{minipage}{\textwidth}\bf
	% FIXME write your abstract here
	
	This is an abstract.
\end{minipage}


% In setting up this template for *Science* papers, we've used both
% the \section* command and the \paragraph* command for topical
% divisions.  Which you use will of course depend on the type of paper
% you're writing.  Review Articles tend to have displayed headings, for
% which \section* is more appropriate; Research Articles, when they have
% formal topical divisions at all, tend to signal them with bold text
% that runs into the paragraph, for which \paragraph* is the right
% choice.  Either way, use the asterisk (*) modifier, as shown, to
% suppress numbering.


% FIXME if you want two columns after the abstract, remove the %
% you also have to remove the leading % before \end{multicols}
% at the end of this file
% the 2 specifies the number of columns
%\begin{multicols}{2}

\section*{Introduction}

\lipsum


\section*{Trapattonis Rede von 1998}

Es gibt im Moment in diese Mannschaft, oh, einige Spieler vergessen ihnen Profi was sie sind. Ich lese nicht sehr viele Zeitungen, aber ich habe gehört viele Situationen. Erstens: wir haben nicht offensiv gespielt. Es gibt keine deutsche Mannschaft spielt offensiv und die Name offensiv wie Bayern. Letzte Spiel hatten wir in Platz drei Spitzen: Elber, Jancka und dann Zickler. Wir müssen nicht vergessen Zickler. Zickler ist eine Spitzen mehr, Mehmet eh mehr Basler. Ist klar diese Wörter, ist möglich verstehen, was ich hab gesagt? Danke. Offensiv, offensiv ist wie machen wir in Platz. Zweitens: ich habe erklärt mit diese zwei Spieler: nach Dortmund brauchen vielleicht Halbzeit Pause. Ich habe auch andere Mannschaften gesehen in Europa nach diese Mittwoch. Ich habe gesehen auch zwei Tage die Training. Ein Trainer ist nicht ein Idiot! Ein Trainer sei sehen was passieren in Platz. In diese Spiel es waren zwei, drei diese Spieler waren schwach wie eine Flasche leer! Haben Sie gesehen Mittwoch, welche Mannschaft hat gespielt Mittwoch? Hat gespielt Mehmet oder gespielt Basler oder hat gespielt Trapattoni? Diese Spieler beklagen mehr als sie spielen! Wissen Sie, warum die Italienmannschaften kaufen nicht diese Spieler? Weil wir haben gesehen viele Male solche Spiel! \cite{Bottou2014}


\section*{Lorem Ipsum Alternativen}

Das hier ist nur sinnloser Text um am Ende eine andere Zitationsart darstellen zu können, wie schon \citealp{Wartena2015} darlegten.

\paragraph*{Paragraph Header.} So sieht ein \textit{Paragraph} in diesem Dokument aus. 
Die Überschrift ist auf einer Ebene wie die erste Textzeile -- anders als bei \textit{section}.

Nachfolgend noch einer.

\paragraph*{Samuel L. Ipsum.}  Normally, both your asses would be dead as fucking fried chicken, but you happen to pull this shit while I'm in a transitional period so I don't wanna kill you, I wanna help you. But I can't give you this case, it don't belong to me. Besides, I've already been through too much shit this morning over this case to hand it over to your dumb ass.

Your bones don't break, mine do. That's clear. Your cells react to bacteria and viruses differently than mine. You don't get sick, I do. That's also clear. But for some reason, you and I react the exact same way to water. We swallow it too fast, we choke. We get some in our lungs, we drown. However unreal it may seem, we are connected, you and I. We're on the same curve, just on opposite ends.
\url{http://slipsum.com/}


% Path to to the bibliography file - extension is .bib and must not be added in the path
\bibliography{basics/bibliography}
% Choosen bibliography style from natbib package
\bibliographystyle{abbrvnat}

% FIXME if you applied the two column multicols style, you need to remove the %
%\end{multicols}

\end{document}