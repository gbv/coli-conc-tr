\documentclass[12pt,        % standard font size
  english,ngerman,          % english primary, german secondary
  paper=a4,                 % standard paper size
  captions=tablesignature,  % captions below tables
  listof=numbered,          % including lists of... in the ToC
  bibliography=totoc,       % references in ToC
  headings=small,           % size of headings
  headinclude=false,        % don't include page head in page layout
  footinclude=false,        % don't include page foot in page layout
  parskip=half-,            % space between paragraphs, no indentation
%   dotlessnumbers,         % no dots after chapter etc. numbers
% FIXME Adjust following 2 options for two-sided print, nr of pages
  oneside,                  % one-sided print
%   twoside,                % two-sided print
% FIXME Adjust the Binding CORrection to the binding and number of pages
% 15mm should be fine for theses with about 60 to 120 double sided pages
% and with the standard binding for theses (hardcover)
% Add 2-3mm for longer theses, substract 2-3 mm for shorter theses
% For more information see the documentation for KOMA-classes
%  BCOR=15mm,                 % BCOR for thesis (hardcover)
%  BCOR=5mm,                 % BCOR for project reports (plastic folder)
%                             % No BCOR for seminar reports (plastic folder)
%   draft,                  % speed things up before final version
%   DIV=calc                % calculate the page layout
  DIV=12                    %  
%   DIV=classic             %
  ]{styles/colidok}


% most setup is done in a file of its own
%%%%%%%%%%%%%%%%%%%%%%%%%%%%%%%%%%%%%%%%%%%%%%%%%%%%%%%%%%%%
% Now we can start to load the packages needed
%%%%%%%%%%%%%%%%%%%%%%%%%%%%%%%%%%%%%%%%%%%%%%%%%%%%%%%%%%%%

%%%%%%%%%% Testing  %%%%%%%%%%
% provides Lorem Ipsum blindtext
\usepackage{lipsum}


%%%%%%%%%% General      %%%%%%%%%%
% koma comes with more advanced handling of tocs, scrhack
% makes other packages like listings work with that
% TODO this might cause problems with updated packages
\usepackage{scrhack}
% we want to support both pdfLaTeX and XeLaTeX
% TODO consider luaTeX as well
\usepackage{ifxetex}
% language settings, output-encoding
% localisation (languages set in document directive):
\usepackage{babel}
% % TODO context sensitive quotes, investigate for biblatex
% \usepackage{csquotes}
% input encoding to support diacritics
\usepackage[utf8]{inputenc}
% output encoding:
\usepackage[T1]{fontenc}
% acronym list
\usepackage[printonlyused]{acronym}


%%%%%%%%%% Look & Feel  %%%%%%%%%%
% define colors as you need them
\usepackage{color}
% nicer headings/footer:
\usepackage{scrpage2}
% typeset URLs nicely:
\usepackage[hyphens]{url}
% extended table environment:
\usepackage{tabularx}
% % colours in tables:
% \usepackage{colortbl}
% % booktabs for nice looking tables
% \usepackage{booktabs}
% used for adapting list environment, e.g. for glossaries:
\usepackage{enumitem}
% include source code listings in the text:
\usepackage{listings}

\usepackage{multicol}

%%%%%%%%%% Graphics     %%%%%%%%%%
% we need to be able to include graphics:
\usepackage{graphicx}
% we use colours in tables etc.:
\usepackage[dvipsnames,usenames]{xcolor}
% TikZ is cool for diagrams:
\usepackage{tikz}
% these 2 packages can be used to generate
% graphs & tables from tabular data:
\usepackage{pgfplots}
\usepackage{pgfplotstable}
% pinning to specific pgfplots version
\pgfplotsset{compat=1.8}
% make text flow around images
% \usepackage{picins}

%%%%%%%%%% Color Definitions %%%%%
% color definitions for listings (example in java)
\definecolor{pblue}{rgb}{0.13,0.13,1}
\definecolor{pgreen}{rgb}{0,0.5,0}
\definecolor{pred}{rgb}{0.9,0,0}
\definecolor{pgrey}{rgb}{0.46,0.45,0.48}

% color definitions corporate design vzg and coli-conc
\definecolor{vzgheadline}{RGB}{0,102,102}
\definecolor{vzgpetrol}{RGB}{47,140,144}
\definecolor{vzgblue}{RGB}{84,128,187}
\definecolor{coliorange}{HTML}{C14E00}
\definecolor{coliblack}{HTML}{393939}
\definecolor{coligrey}{HTML}{E3E3DF}

%%%%%%%%%% Bib          %%%%%%%%%%
% allows us to use author-year styles:
\usepackage[round,colon,sort]{natbib}
% \usepackage{natbib}
% allows to have a separate bibliography for weblinks (or more):
\usepackage{multibib}

%%%%%%%%%% Other        %%%%%%%%%%
% % is already loaded automatically:
% \usepackage{calc}
% % do we want an index?
% \usepackage{makeidx}
% include hyperlinks in PDF and much more:
% FIXME Chose between coloured or black
\usepackage[
  colorlinks=true,
  linkcolor=coliorange,
  menucolor=coliorange,
  citecolor=coliorange,
  urlcolor=coliorange,
]{hyperref}
\hypersetup{linktoc=all}

%%%%%%%%%%%%%%%%%%%%%%%%%%%%%%%%%%%%%%%%%%%%%%%%%%%%%%%%%%%%
%   lists of... and captions
%%%%%%%%%%%%%%%%%%%%%%%%%%%%%%%%%%%%%%%%%%%%%%%%%%%%%%%%%%%%

% for listings, we define the names used in the captions
% and in the list of listings
% we also take the -verzeichnis out of the list of
% figures and list of tables
% TODO differences in defined languages/babel; Text formating
\AtBeginDocument{%
  \makeatletter
    \addto\captionsngerman{%
      \def\lstlistingname{Quelltext}%
      \def\lstlistlistingname{Quelltexte}%
%       \renewcommand{\listfigurename}{\normalfont\rmfamily Abbildungen}%
%       \renewcommand{\listtablename}{\normalfont\rmfamily Tabellen}%
      \renewcommand{\listfigurename}{Abbildungen}%
      \renewcommand{\listtablename}{Tabellen}%
    }%
  \makeatother
}
% % for other than KOMA classes, the following may have
% % to be used instead:
% \renewcommand*{\lstlistlistingname}{Quelltexte}
% \renewcommand*{\lstlistingname}{Quelltexte}

% captions should be set in a small font size
\addtokomafont{caption}{\footnotesize}
% \addtokomafont{captionlabel}{\bfseries}

%%%%%%%%%%%%%%%%%%%%%%%%%%%%%%%%%%%%%%%%%%%%%%%%%%%%%%%%%%%%
%   General changes to formatting
%%%%%%%%%%%%%%%%%%%%%%%%%%%%%%%%%%%%%%%%%%%%%%%%%%%%%%%%%%%%

% this is the KOMA-way of changing titles.
%\setkomafont{sectioning}{\rmfamily\bfseries}
\setkomafont{disposition}{\color{coliorange}\rmfamily\bfseries}
  
% we want an url to be printed in the standard font
\urlstyle{same}
% additional setup specifications for the documentation
% this makes ``References'' a section instead of a chapter
% this has to be done before other packages mess around
% with the definition
\makeatletter
\renewcommand*\bib@heading{%
	\section*{\bibname}%
	\@mkboth{\bibname}{\bibname}%
}

% Change colour of superscriptes numbers from footnotes
\renewcommand\@makefnmark{\hbox{
  \@textsuperscript{\normalfont\color{coliorange}\@thefnmark}}
}

\makeatother

%%%%%%%%%%%%%%%%%%%%%%%%%%%%%%%%%%%%%%%%%%%%%%%%%%%%%%%%%%%%
% Set up header and footer
%%%%%%%%%%%%%%%%%%%%%%%%%%%%%%%%%%%%%%%%%%%%%%%%%%%%%%%%%%%%

\pagestyle{scrheadings}
\clearscrheadfoot
\renewcommand{\headfont}{\normalfont\rmfamily\itshape}
\ohead{\headmark}
% FIXME choose between english and german version
\lofoot[Coli-conc Report No. n]{Coli-conc Report No. n}
% \lofoot[Coli-conc Report Nr. n]{Coli-conc Report Nr. n}
\ofoot[\pagemark]{\pagemark}


%%%%%%%%%%%%%%%%%%%%%%%%%%%%%%%%%%%%%%%%%%%%%%%%%%%%%%%%%%%%
% Set up citation styles
%%%%%%%%%%%%%%%%%%%%%%%%%%%%%%%%%%%%%%%%%%%%%%%%%%%%%%%%%%%%

% we use multibib.sty to generate a weblinks bibliography,
% define more as you see fit (Software, RFCs, Images)
\newcites{web}{Weblinks}
% % FIXME If you use several bibliographies, you need to run
% % all of them through bibTeX. This can be achieved with the
% % following script (for Unix, Mac, and Linux):
% #!/bin/bash
% for file in *.aux ; do
%   bibtex `basename $file .aux`
% done

%%%%%%%%%%%%%%%%%%%%%%%%%%%%%%%%%%%%%%%%%%%%%%%%%%%%%%%%%%%%
% float control
%%%%%%%%%%%%%%%%%%%%%%%%%%%%%%%%%%%%%%%%%%%%%%%%%%%%%%%%%%%%
% more sensible defaults for how much space floats can take
% up on pages with text:
% http://www.tug.org/texmf-dist/doc/generic/FAQ-en/html/FAQ-floats.html
\setcounter{topnumber}{2}
\setcounter{bottomnumber}{9}
\setcounter{totalnumber}{20}
\setcounter{dbltopnumber}{9}
\renewcommand{\topfraction}{0.85}
\renewcommand{\bottomfraction}{0.7}
\renewcommand{\textfraction}{0.15}
\renewcommand{\floatpagefraction}{0.7}
\renewcommand{\dbltopfraction}{.7}
\renewcommand{\dblfloatpagefraction}{.7}
% in the following file all necessary input has to be made
%%%%%%%%%%%%%%%%%%%%%%%%%%%%%%%%%%%%%%%%%%%%%%%%%%%%%%%%%%%%
%   General information
%%%%%%%%%%%%%%%%%%%%%%%%%%%%%%%%%%%%%%%%%%%%%%%%%%%%%%%%%%%%

% FIXME enter document title
\newcommand{\colititle}{Lorem Ipsum Titelum}

% FIXME enter the subject(s) divided by comma
\newcommand{\colisubject}{Template, VZG, Coliconc}

% FIXME real date of delivery. \today is todays date as dd. MM yyyy
% \newcommand{\colidate}{\today}
\newcommand{\colidate}{1. April 1970}


%%%%%%%%%%%%%%%%%%%%%%%%%%%%%%%%%%%%%%%%%%%%%%%%%%%%%%%%%%%%
%   For coli-documentation only
%%%%%%%%%%%%%%%%%%%%%%%%%%%%%%%%%%%%%%%%%%%%%%%%%%%%%%%%%%%%

% FIXME enter the document author(s)
\newcommand{\coliauthor}{Mustermann, Max und Musterfrau, Martha}

% FIXME (un)mark the needed text type
\newcommand{\colitexttype}{Technical Documentation}
% \newcommand{\colitexttype}{Technische Dokumentation}

% FIXME (un)mark the version that is needed
\newcommand{\colireportlang}{No. }
% \newcommand{\colireportlang}{Nr. }

% FIXME enter the report number
\newcommand{\colireportno}{n}


%%%%%%%%%%%%%%%%%%%%%%%%%%%%%%%%%%%%%%%%%%%%%%%%%%%%%%%%%%%%
%   For coli-article only
%%%%%%%%%%%%%%%%%%%%%%%%%%%%%%%%%%%%%%%%%%%%%%%%%%%%%%%%%%%%

% Number of authors and working addresses has to correspond.
% The tex-file has to be adjusted accordingly.
% FIXME enter the document author(s)
\newcommand{\coliauthorone}{Mustermann, Max}
\newcommand{\coliauthortwo}{Musterfrau, Martha}
\newcommand{\coliauthorthree}{Bla}
\newcommand{\coliauthorfour}{Blubb}

% FIXME enter the author(s) working address(es)
\newcommand{\coliaddressone}{Department of Chemistry, University of Wherever, Street, Post Code, City}
\newcommand{\coliaddresstwo}{Department of Computer Science, University of Wherever}
\newcommand{\coliaddressthree}{Department of Biology, University of Wherever}
\newcommand{\coliaddressfour}{Department of Dingsda, University of Wherever}

% FIXME enter correspondence email address
\newcommand{\coliemail}{musterdingens@gbv.de}

% FIXME enter abstract
\newcommand{\coliabstract}{
	Jemand musste Josef K. verleumdet haben, denn ohne dass er etwas Böses getan hätte, wurde er eines Morgens verhaftet. "Wie ein Hund!" sagte er, es war, als sollte die Scham ihn überleben. Als Gregor Samsa eines Morgens aus unruhigen Träumen erwachte, fand er sich in seinem Bett zu einem ungeheueren Ungeziefer verwandelt. Und es war ihnen wie eine Bestätigung ihrer neuen Träume und guten Absichten, als am Ziele ihrer Fahrt die Tochter als erste sich erhob und ihren jungen Körper dehnte. "Es ist ein eigentümlicher Apparat", sagte der Offizier zu dem Forschungsreisenden und überblickte mit einem gewissermaßen bewundernden Blick den ihm doch wohlbekannten Apparat. Sie hätten noch ins Boot springen können, aber der Reisende hob ein schweres, geknotetes Tau vom Boden, drohte ihnen damit und hielt sie dadurch von dem Sprunge ab. In den letzten Jahrzehnten ist das Interesse an Hungerkünstlern sehr zurückgegangen. Aber sie überwanden sich, umdrängten den Käfig und wollten sich gar nicht fortrühren.
}

% do we want to include a glossary and an index?
% \makeglossary
\makeindex


%%%%%%%%%%%%%%%%%%%%%%%%%%%%%%%%%%%%%%%%%%%%%%%%%%%%%%%%%%%%
% Start with the document
%%%%%%%%%%%%%%%%%%%%%%%%%%%%%%%%%%%%%%%%%%%%%%%%%%%%%%%%%%%%

\begin{document}

% stuff before real content, no chapter marks, roman page numbers
\frontmatter
\pagenumbering{Roman}

% % the cite commands will generate entries in the index, if used
% \citeindextrue

%%%%%%%%%%%%%%%%%%%%%%%%%%%%%%%%%%%%%%%%%%%%%%%%%%%%%%%%%%%%
% We don't use a custom \maketitle for now
% so we have to make our own title page
%%%%%%%%%%%%%%%%%%%%%%%%%%%%%%%%%%%%%%%%%%%%%%%%%%%%%%%%%%%%

\begin{titlepage}
  \hypersetup{%
    pdftitle={\colititle},
    pdfauthor={\coliauthor},
    pdfsubject={\colisubject}
  }
 
\par\medskip
\noindent\makebox[\linewidth][c]{%
  \colorbox{vzgblue}{%
      \parbox[l][1.2em][c]{\paperwidth}{%
          \hspace*{\dimexpr\hoffset+\oddsidemargin+1in\relax}%
        \begin{minipage}{.3\textwidth}
            \includegraphics[width=\linewidth, height=1.7em, keepaspectratio]{basics/vzg.png}
        \end{minipage}%
        \begin{minipage}{.05\textwidth}
            \mbox{}
        \end{minipage}%
        \begin{minipage}{.7\textwidth}
          \begin{flushright}
            \textbf{{\footnotesize \textcolor{white}{Verbundzentrale des GBV}}}\relax
          \end{flushright}
      \end{minipage}%
      }%
  }%
}\par\medskip
  
  \begin{center}

    \vfill

    {
% FIXME select used language    
      \selectlanguage{english} % switching to english
      % \selectlanguage{ngerman} % switching to german

      \Huge
      \bfseries
      \textcolor{vzgblue}{
      \colitexttype
    }
    
  \Large
  \bfseries
  \textcolor{vzgblue}{
  Report \colireportlang \colireportno
  }
      
      \vspace{1.25cm}
      
      \huge
      \mdseries
        \colititle
    }

    \vfill

    {
      \normalsize

      \vfill
      \vfill
      
      Göttingen, \colidate
    }
      
      \vfill
      
    {

    \includegraphics[scale=0.3]{basics/coli-conc-wave.png}

    }
  \end{center}
\end{titlepage}


% if using tocloft.sty, we need an explicit clearpage
\clearpage
% we want chapters and sections in the toc, then generate it
\setcounter{tocdepth}{1}

\tableofcontents

% now we have real content, with section marks and egyptian numerals
\mainmatter

%%%%%%%%%%%%%%%%%%%%%%%%%%%%%%%%%%%%%%%%%%%%%%%%%%%%%%%%%%%%
% WHERE THE REAL WORK TAKES PLACE
%%%%%%%%%%%%%%%%%%%%%%%%%%%%%%%%%%%%%%%%%%%%%%%%%%%%%%%%%%%%
\selectlanguage{english} % switching to english
% \selectlanguage{ngerman} % switching to german

%%%%%%%%%%%%%%%%%%%%%%%%%%%%%%%%%%%%%%%%%%%%%%%%%%%%%%%%%%%%
\chapter{Einleitung}\label{chapter:einleitung}
%%%%%%%%%%%%%%%%%%%%%%%%%%%%%%%%%%%%%%%%%%%%%%%%%%%%%%%%%%%%

%  Einführung und Motivation des Themas
%  Kurze Einleitung zu den Unterkapiteln

Do you see any Teletubbies in here? Do you see a slender plastic tag clipped to my shirt with my name printed on it? Do you see a little Asian child with a blank expression on his face sitting outside on a mechanical helicopter that shakes when you put quarters in it? No? Well, that's what you see at a toy store. And you must think you're in a toy store, because you're here shopping for an infant named Jeb.\footnote{\citet{Wartena2015}}

\newpage

%%%%%%%%%%%%%%%%%%%%%%%%%%%%%%%%%%%%%%%%%%%%%%%%%%%%%%%%%%%%
\section{Page Two}\label{sec:page_two}
%%%%%%%%%%%%%%%%%%%%%%%%%%%%%%%%%%%%%%%%%%%%%%%%%%%%%%%%%%%%

Jemand musste Josef K. verleumdet haben, denn ohne dass er etwas Böses getan hätte, wurde er eines Morgens verhaftet. "Wie ein Hund!" sagte er, es war, als sollte die Scham ihn überleben. Als Gregor Samsa eines Morgens aus unruhigen Träumen erwachte, fand er sich in seinem Bett zu einem ungeheueren Ungeziefer verwandelt. Und es war ihnen wie eine Bestätigung ihrer neuen Träume und guten Absichten, als am Ziele ihrer Fahrt die Tochter als erste sich erhob und ihren jungen Körper dehnte. "Es ist ein eigentümlicher Apparat", sagte der Offizier zu dem Forschungsreisenden und überblickte mit einem gewissermaßen bewundernden Blick den ihm doch wohlbekannten Apparat. Sie hätten noch ins Boot springen können, aber der Reisende hob ein schweres, geknotetes Tau vom Boden, drohte ihnen damit und hielt sie dadurch von dem Sprunge ab. In den letzten Jahrzehnten ist das Interesse an Hungerkünstlern sehr zurückgegangen. Aber sie überwanden sich, umdrängten den Käfig und wollten sich gar nicht fortrühren.\footnote{\citet{Bottou2014}}

Dieser Text stammt von dem Herren in Bild \ref{fig:kafka}.

\begin{figure}[tbph]
	\centering
	\includegraphics[width=0.7\linewidth]{./figures/kafka}
	\caption[Kafka]{Portrait von Franz Kafka}
	\label{fig:kafka}
\end{figure}


\newpage

%%%%%%%%%%%%%%%%%%%%%%%%%%%%%%%%%%%%%%%%%%%%%%%%%%%%%%%%%%%%
\section{Page Three}\label{sec:page_three}
%%%%%%%%%%%%%%%%%%%%%%%%%%%%%%%%%%%%%%%%%%%%%%%%%%%%%%%%%%%%

For those who have seen the Earth from space, and for the hundreds and perhaps thousands more who will, the experience most certainly changes your perspective. The things that we share in our world are far more valuable than those which divide us.

Buy why, some say, the moon? Why choose this as our goal? And they may as well ask why climb the highest mountain?

Curious that we spend more time congratulating people who have succeeded than encouraging people who have not.

Across the sea of space, the stars are other suns.

When I orbited the Earth in a spaceship, I saw for the first time how beautiful our planet is. Mankind, let us preserve and increase this beauty, and not destroy it!


%%%%%%%%%%%%%%%%%%%%%%%%%%%%%%%%%%%%%%%%%%%%%%%%%%%%%%%%%%%%
\chapter{Page Four}\label{sec:page_four}
%%%%%%%%%%%%%%%%%%%%%%%%%%%%%%%%%%%%%%%%%%%%%%%%%%%%%%%%%%%%

Let him go! Stay on the leader! Hurry, Luke, they're coming in much faster this time. I can't hold them! Artoo, try and increase the power! Hurry up, Luke! Wait! I'm on the leader. Hang on, Artoo! Use the Force, Luke. Let go, Luke. The Force is strong with this one! Luke, trust me. His computer's off. Luke, you switched off your targeting computer. What's wrong? Nothing. I'm all right. I've lost Artoo! You may fire when ready. I have you now.

He says it's the best he can do. Since the XP-38 came out, they're just not in demand. It will be enough. If the ship's as fast as he's boasting, we ought to do well.

To your stations! Come with me. Close all outboard shields! Close all outboard shields! Yes. We've captured a freighter entering the remains of the Alderaan system. It's markings match those of a ship that blasted its way out of Mos Eisley. They must be trying to return the stolen plans to the princess. She may yet be of some use to us. Unlock one-five-seven and nine. Release charges. There's no one on board, sir. According to the log, the crew abandoned ship right after takeoff. It must be a decoy, sir. Several of the escape pods have been jettisoned.

Luke! Luke! Luke! Hey! Hey! I knew you'd come back! I just knew it! Well, I wasn't gonna let you get all the credit and take all the reward. Hey, I knew there was more to you than money. Oh, no! Oh, my! Artoo! Can you hear me? Say something! You can repair him, can't you? We'll get to work on him right away. You must repair him! Sir, if any of my circuits or gears will help, I'll gladly donate them. He'll be all right.

Your friend is quite a mercenary. I wonder if he really cares about anything...or anyone. I care! So...what do you think of her, Han? I'm trying not to, kid! Good... Still, she's got a lot of spirit. I don't know, what do you think? Do you think a princess and a guy like me... No!

Man könnte auch mal Tabelle \ref{tab:threecols} referenzieren.

\begin{table}
	\centering
	\begin{tabular}{lll}
		\textbf{Überschrift 1} & \textbf{Überschrift 2} & \textbf{Überschrift 3} \\
		A & B & C \\
		1 & 2 & 3 \\
		a & b & c \\
		i & ii & iii \\
	\end{tabular}
	\caption{Eine dreispaltige Tabelle}
	\label{tab:threecols}
\end{table}


%%%%%%%%%%%%%%%%%%%%%%%%%%%%%%%%%%%%%%%%%%%%%%%%%%%%%%%%%%%%
\section{Page Five}\label{sec:page_five}
%%%%%%%%%%%%%%%%%%%%%%%%%%%%%%%%%%%%%%%%%%%%%%%%%%%%%%%%%%%%

Clueless ghost pottery scene frosted tips tommy hilfiger. Snapple keepin’ it real bare midriffs punk, mariah carey life is like a box of chocolates butterfly clips atlanta summer olympics tae bo. I'm king of the world millenials track jackets wild cherry pepsi home improvement. As if quiet storm crop tops once you pop you can’t stop furby, zack morris scrolling text headbands snapback hats natalie imbruglia teenage mutant ninja turtles. Christina aguilera fila hi-top fade chia pet tim “the tool man” taylor.

Listing \ref{lst:java} lässt sich auch referenzieren.

Keepin’ it real internet screening phone calls george clooney ac slater rad. Highlights backstreet boys honda accord mr. jones and me tell each other fairy tales, home alone end of the road buddy list sports utility vehicles choker necklace schindler’s list. Stonewashed blue jeans hip hop central perk sonic the hedgehog dial-up the truman show. Men in black grunge I will always love you rollerblades videocassette free willy. Bubble tape apollo 13 charlotte hornets renting movies at a store fresh roseanne barr.


\begin{lstlisting}[language=Java, caption=Java example, label=lst:java]
class Hallo {
  public static void main( String[] args ) {
    System.out.print("Hallo Welt!");
  }
}
\end{lstlisting}



%%%%%%%%%%%%%%%%%%%%%%%%%%%%%%%%%%%%%%%%%%%%%%%%%%%%%%%%%%%%
% list of figures, tables
%%%%%%%%%%%%%%%%%%%%%%%%%%%%%%%%%%%%%%%%%%%%%%%%%%%%%%%%%%%%

% here we have all the stuff where chapters have no numbers etc.
% you will find a lot of \cleardoublepage and \phantomsection
% commands, these help hyperref.sty to find the right targets
% for hyperlinks
\backmatter

% The list of figures
  \cleardoublepage
  \phantomsection
%   \addcontentsline{toc}{chapter}{Abbildungen}
  \listoffigures

% The list of tables
  \cleardoublepage
  \phantomsection
%   \addcontentsline{toc}{chapter}{Tabellen}
  \listoftables

% list of listings
  \cleardoublepage
  \phantomsection
%   \addcontentsline{toc}{chapter}{Quelltexte}
  \lstlistoflistings  % generated by listings.sty

%%%%%%%%%%%%%%%%%%%%%%%%%%%%%%%%%%%%%%%%%%%%%%%%%%%%%%%%%%%%
%%%%%%%%%%%%%%%%%%%%%%%%%%%%%%%%%%%%%%%%%%%%%%%%%%%%%%%%%%%%
\chapter{Quellen}
% \addcontentsline{toc}{chapter}{Quellen}
%%%%%%%%%%%%%%%%%%%%%%%%%%%%%%%%%%%%%%%%%%%%%%%%%%%%%%%%%%%%

% While it is possible to cite something without putting
% in a reference, the question is, why would you do that?
% \nociteweb{imai}

\phantomsection
\renewcommand{\bibname}{Literatur}
\bibliography{basics/bibliography}
\bibliographystyle{abbrvnat}
\addcontentsline{toc}{section}{Literatur}

\phantomsection
\bibliographystyleweb{imai}
\bibliographyweb{basics/bibliography}
\addcontentsline{toc}{section}{Weblinks}


\section*{Software}
\addcontentsline{toc}{section}{Software}

\LaTeX, 08. März 2012, \url{http://www.latex-project.org/}

openSUSE 12.1, 08. März 2012, \url{http://www.opensuse.org/}

%%%%%%%%%%%%%%%%%%%%%%%%%%%%%%%%%%%%%%%%%%%%%%%%%%%%%%%%%%%%
% we now go into the appendix. basically the appendix is one chapter
% with some sections below. The appendix chapter has no mark, the
% sections alphabetical ones, subsections will have numerical ones
% TODO format lower level sections in appendix
\renewcommand{\thechapter}{\Alph{chapter}}
\renewcommand{\thesection}{\Alph{section}}
\renewcommand{\thesubsection}{\Alph{section}.\arabic{subsection}}
\setcounter{section}{0} % need to be explicit, since they are not reset in back matter
\addtocounter{chapter}{1} % help hyperref find the correct section

%%%%%%%%%%%%%%%%%%%%%%%%%%%%%%%%%%%%%%%%%%%%%%%%%%%%%%%%%%%%
%%%%%%%%%%%%%%%%%%%%%%%%%%%%%%%%%%%%%%%%%%%%%%%%%%%%%%%%%%%%
\chapter{Anhänge}\label{chapter:appendix}
% \addcontentsline{toc}{chapter}{Anhänge}
%%%%%%%%%%%%%%%%%%%%%%%%%%%%%%%%%%%%%%%%%%%%%%%%%%%%%%%%%%%%

%  Umfangreiche zusätzliche Informationen, die im Textverlauf stören würden, die aber für 
%  die Arbeit wichtig sind, wie z.B. Programmcode, Fragebögen, Evaluationstabellen

%%%%%%%%%%%%%%%%%%%%%%%%%%%%%%%%%%%%%%%%%%%%%%%%%%%%%%%%%%%%
\section{Programmcode}\label{sec:appendix_code}
%%%%%%%%%%%%%%%%%%%%%%%%%%%%%%%%%%%%%%%%%%%%%%%%%%%%%%%%%%%%

%  Es geht hier nicht darum, den kompletten Quelltext des praktischen Teils der Arbeit 
%  abzudrucken
  
%  Es sollen lediglich besonders wichtige Fragmente (API-Definitionen, 
%  Kommunikationsprotokolle) dokumentiert werden


\end{document}