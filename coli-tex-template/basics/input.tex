%%%%%%%%%%%%%%%%%%%%%%%%%%%%%%%%%%%%%%%%%%%%%%%%%%%%%%%%%%%%
%   General information
%%%%%%%%%%%%%%%%%%%%%%%%%%%%%%%%%%%%%%%%%%%%%%%%%%%%%%%%%%%%

% FIXME enter document title
\newcommand{\colititle}{Lorem Ipsum Titelum}

% FIXME enter the subject(s) divided by comma
\newcommand{\colisubject}{Template, VZG, Coliconc}

% FIXME real date of delivery. \today is todays date as dd. MM yyyy
% \newcommand{\colidate}{\today}
\newcommand{\colidate}{1. April 1970}


%%%%%%%%%%%%%%%%%%%%%%%%%%%%%%%%%%%%%%%%%%%%%%%%%%%%%%%%%%%%
%   For coli-documentation only
%%%%%%%%%%%%%%%%%%%%%%%%%%%%%%%%%%%%%%%%%%%%%%%%%%%%%%%%%%%%

% FIXME enter the document author(s)
\newcommand{\coliauthor}{Mustermann, Max und Musterfrau, Martha}

% FIXME (un)mark the needed text type
\newcommand{\colitexttype}{Technical Documentation}
% \newcommand{\colitexttype}{Technische Dokumentation}

% FIXME (un)mark the version that is needed
\newcommand{\colireportlang}{No. }
% \newcommand{\colireportlang}{Nr. }

% FIXME enter the report number
\newcommand{\colireportno}{n}


%%%%%%%%%%%%%%%%%%%%%%%%%%%%%%%%%%%%%%%%%%%%%%%%%%%%%%%%%%%%
%   For coli-article only
%%%%%%%%%%%%%%%%%%%%%%%%%%%%%%%%%%%%%%%%%%%%%%%%%%%%%%%%%%%%

% Number of authors and working addresses has to correspond.
% The tex-file has to be adjusted accordingly.
% FIXME enter the document author(s)
\newcommand{\coliauthorone}{Mustermann, Max}
\newcommand{\coliauthortwo}{Musterfrau, Martha}
\newcommand{\coliauthorthree}{Bla}
\newcommand{\coliauthorfour}{Blubb}

% FIXME enter the author(s) working address(es)
\newcommand{\coliaddressone}{Department of Chemistry, University of Wherever, Street, Post Code, City}
\newcommand{\coliaddresstwo}{Department of Computer Science, University of Wherever}
\newcommand{\coliaddressthree}{Department of Biology, University of Wherever}
\newcommand{\coliaddressfour}{Department of Dingsda, University of Wherever}

% FIXME enter correspondence email address
\newcommand{\coliemail}{musterdingens@gbv.de}

% FIXME enter abstract
\newcommand{\coliabstract}{
	Jemand musste Josef K. verleumdet haben, denn ohne dass er etwas Böses getan hätte, wurde er eines Morgens verhaftet. "Wie ein Hund!" sagte er, es war, als sollte die Scham ihn überleben. Als Gregor Samsa eines Morgens aus unruhigen Träumen erwachte, fand er sich in seinem Bett zu einem ungeheueren Ungeziefer verwandelt. Und es war ihnen wie eine Bestätigung ihrer neuen Träume und guten Absichten, als am Ziele ihrer Fahrt die Tochter als erste sich erhob und ihren jungen Körper dehnte. "Es ist ein eigentümlicher Apparat", sagte der Offizier zu dem Forschungsreisenden und überblickte mit einem gewissermaßen bewundernden Blick den ihm doch wohlbekannten Apparat. Sie hätten noch ins Boot springen können, aber der Reisende hob ein schweres, geknotetes Tau vom Boden, drohte ihnen damit und hielt sie dadurch von dem Sprunge ab. In den letzten Jahrzehnten ist das Interesse an Hungerkünstlern sehr zurückgegangen. Aber sie überwanden sich, umdrängten den Käfig und wollten sich gar nicht fortrühren.
}